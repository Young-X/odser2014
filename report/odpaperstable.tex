\begin{landscape}
\begin{longtabu} {X[3]XXXXX[2]}
\caption{Papers that mention outliers and outlier detection. \emph{Used in experiment} refers to if the paper was part of the application of outliers in Section~\ref{sec:results-application}} \\
\hline
\textbf{Paper\strut} & \textbf{Mention outliers\strut} & \textbf{Outlier detection\strut} & \textbf{Used in experiment\strut} & \textbf{Reference\strut} & \textbf{Comment\strut} \\ \hline \endhead

Are test cases needed? Replicated comparison between exploratory and test-case-based software testing\strut                                                     & Yes                       & Yes & Yes                      & \citet{itkonen2013test}          & Finds outliers and mentions that they include the outliers in the analysis.\strut
\\ \hline

Adoption and use of Java generics\strut                                                                                                                         & Yes                       & No & Yes                         & \citet{parnin2013adoption}       & Mentions outlier but does not describe how they were detected or if they exclude or include them.\strut
\\ \hline

Effectiveness for detecting faults within and outside the scope of testing techniques: an independent replication\strut                                         & Yes                       & No & No                         & \citet{apaeffectiveness}         & Does not mention if they include or exclude the outliers found.\strut
\\ \hline

A replicated quasi-experimental study on the influence of personality and team climate in software development\strut                                            & Yes                       & No & No                         & \citet{gomez2013replicated}      & Does not mention how outlier detection is conducted just mentions that the ``analyst'' is responsible for identifying possible outliers.\strut
\\ \hline

Using tabu search to configure support vector regression for effort estimation\strut                                                                            & Yes                       & No & No                        & \citet{corazza2013using}         & Mention that outliers can be a problem problems in some cases.\strut
\\ \hline

Do background colors improve program comprehension in the \#ifdef hell?\strut                                                                                   & Yes                       & No & Yes                        & \citet{feigenspan2013background} & Uses a box-plot but does not say if they remove the data points or not.\strut
\\ \hline

Can traditional fault prediction models be used for vulnerability prediction?\strut                                                                             & Yes                       & No & No                        & \citet{shin2013can}              & Mentions that they found outliers, but not with which method.\strut
\\ \hline

Building a second opinion: learning cross-company data\strut                                                                                                    & Yes                       & Yes & No                        & \citet{kocaguneli2013building}   & Describes a reproducible process but does not show which data points they exclude.\strut                      \\ \hline
Effort estimation of FLOSS projects: a study of the Linux kernel\strut                                                                                          & Yes                       & No & No                        & \citet{capiluppi2013effort}      & Determining if a point is an outlier by using manual inspection.\strut
\\ \hline

Mining SQL injection and cross site scripting vulnerabilities using hybrid program analysis\strut                                                               & Yes                       & No & Yes                        & \citet{shar2013mining}           & Uses a clustering algorithm, k-means clustering, but does not present any cutoff value.\strut
\\ \hline

An analysis of multi-objective evolutionary algorithms for training ensemble models based on different performance measures in software effort estimation\strut & Yes                       & Yes & Yes                        & \citet{minku2013analysis}        & References an extended version of the paper where outlier detection is described using k-means clustering.\strut
\\ \hline

Are comprehensive quality models necessary for evaluating software quality?\strut                                                                                & Yes                       & Yes & No                       & \citet{lochmann2013comprehensive}  & Calculates quartiles and presents thresholds.\strut
\\ \hline

The impact of parameter tuning on software effort estimation using learning machines\strut                                                                      & Yes                       & No  & Yes                       & \citet{song2013impact}           & Mentions that outlier detection could be done as future work.\strut
\\ \hline

Human performance regression testing\strut                                                                                                                      & Yes                       & No & No                        & \citet{swearngin2013human}       & Does 5 runs of a experiment in order to lower the probability for any outliers present.\strut
\\ \hline

How, and why, process metrics are better\strut                                                                                                                  & Yes                       & No  & No                       & \citet{rahman2013and}            & They say that they see alot of outliers in the proximity to a box-plot but they do not elaboate on this in the text.\strut
\\ \hline
\end{longtabu}
\end{landscape}
