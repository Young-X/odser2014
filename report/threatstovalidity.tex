\section{Threats to Validity}
\label{sec:threatstovalidity}


\subsection{Internal}
Internal threats to validity involves factors, such as instrumentation settings, that could have affected the outcome.
\begin{itemize}
        \item For conducting outlier detection robust methods were used. These methods used parameters settings that were meant to fit a wide range of data sets and were not specialized. Having used specialized settings for each data set we might discover more\slash less outliers than we did in this study.


        \item Removing a full pair in data sets used for pair-wise testing could be a validity threat as we could potentially remove non-outliers from one data set. This would then make it difficult to conclude anything about the effect of the outlier detection we did initially. However, we did not conduct this ourselves in this study but we like to underline the threat in conducting such elimination of data points.
\end{itemize}


\subsection{External}
External threats to the validity are factors that affect the ability to generalize the results outside the scope of this study.
\begin{itemize}
\item We only collected a current sample from the last one and a half year. This limited sample might not be representable for studies conducted earlier, but we deemed them to be a representative sample of current SE research.
\end{itemize}


\subsection{Conclusion}
Threats to the conclusion validity are factors that might affect how the conclusions are drawn.
\begin{itemize}
\item Some of the data sets gathered during the pre-study did have a small sample size to begin with (see Appendix~\ref{sec:appendix-samplesizes}). The size became even smaller when we removed data points suggested as outliers. The initial small sample size is a validity threat to the original study, but the new smaller size is a validity threat to our study and how we draw conclusions regarding the significant difference of a data set before and after outliers are removed.
\end{itemize}


\subsection{Construct}
Threats to the construct validity can be design errors in the study which could lead to the wrong phenomena being studied. These errors could be caused by social factors.
\begin{itemize}
\item Only having one researcher reviewing each study might have caused a bias. To try and mitigate this risk we discussed issues regarding the studies among us.    
\end{itemize}
