\section{Conclusions}
\label{sec:conclusions}
In this study we have investigated the presence of outliers in ESE studies as well as the effect outliers have on conclusions drawn. In order to conduct this study we needed the original data, documented post-processing and documented analysis as input. Therefore, this study also investigates to what extent the presence of outliers is documented, outlier detection is conducted and data is available from ESE studies.




An automated process was created to conduct outlier analysis and create the results used to answer our hypotheses. This process is available for download\footnote{\url{https://github.com/linqcan/odser2014}} and all results can be reproduced for replication and verification purposes.




Based on the results gathered from the application of an outlier algorithm we can conclude that outliers do exist in ESE studies. 




From the information we collected while preparing our replications we found that 37\% of the investigated studies document outliers in some way and that 25\% out of those do conduct some kind of outlier detection. Regarding the data availability, we found that 26\% of the studies have their data directly available either in the paper or online. Additionally, 12\% of the studies' authors replied with data after we sent out an email request. In total, 37\% of the studies investigated had data available. From this we conclude that the state of replication, in regards to replicating data analysis, is less than desirable within ESE and we think it is in need of improvement.




In order to help the research field improve our study provides the following contributions to the body of knowledge:
\begin{itemize}
        \item Outliers exists within recently published ESE studies and can be found with robust methods.




        \item The extent to which recently published ESE studies document outliers and conduct outlier detection.




        \item The extent to which recently published ESE studies make their data available and how it is made available.




        \item Guidelines for conducting and presenting outlier detection for ESE\@.




        \item Guidelines for how to improve the reproducibility of ESE studies.




        \item An analysis of recently published results and reproducibility within ESE\@.
\end{itemize}
